\documentclass{resume}

\begin{document}

\fontfamily{ppl}\selectfont

\noindent
\begin{tabularx}{\linewidth}{@{}m{0.8\textwidth} m{0.2\textwidth}@{}}
{
    \Large{Александр Смородинов} \newline
    \small{
        \clink{
            \href{mailto:asmorodinov66@gmail.com}{asmorodinov66@gmail.com} \textbf{·} 
            \href{https://github.com/asmorodinov}{github.com/asmorodinov} \textbf{·} \newline
            \href{https://t.me/asmorodinov}{telegram: @asmorodinov}
        } \newline
        Москва, Россия
    }
} & 
{
    \hfill
    \includegraphics[width=2.8cm]{images/gr.png}
}
\end{tabularx}

\begin{center}
\begin{tabularx}{\linewidth}{@{}*{2}{X}@{}}
% left side %
{
    \csection{Опыт работы}{\small
        \begin{itemize}
            % item 1 %
            \item \frcontent{Пока что нулевой}{}{}{}
            
        \end{itemize}
    }
    \csection{Образование}{\small
        \begin{itemize}
            % item 1 %
            \item \frcontent{Студент 2 курса Факультета Компьютерных Наук} {Прикладная Математика и Информатика}{НИУ ВШЭ}{}{2019-2023}
        \end{itemize}
    }
    \csection{Достижения}{\small
        \begin{itemize}
            % item 1 %
            \item \frcontent{Всероссийская олимпиада школьников по математике}{участник заключительного этапа}{}{2019}
            % item 2 %
            \item \frcontent{Турнир Городов (международная математическая олимпиада)}{State of Engineering}{3 диплом}{2019}
            % item 3 %
            \item \frcontent{Московская математическая олимпиада школьников}{2 диплом}{}{2019}
        \end{itemize}
    }
	\csection{Хобби и интересы}{\small
		\begin{itemize}
			\item 3d графика
			\item программирование игр
		\end{itemize}
	}
} 
% end left side %
& 
% right side %
{
    \csection{Навыки}{\small
        \begin{itemize}
            \item \textbf{Технологии} \newline
            {\footnotesize C/C++, GNU assembler x86, Python, Javascript(немного знаю), SQL(тоже немного знаю)}{}{}
            \item \textbf{Библиотеки} \newline
            {\footnotesize GLFW, Bullet physics, entt, assimp}{}{}
        \end{itemize}
    }
    \csection{Проекты}{
        \begin{itemize}
            \item 3d рендерер с нуля \newline \footnotesize \clink{\href{http://github.com/asmorodinov/3d-renderer-from-scratch}{[github.com/asmorodinov/3d-renderer-from-scratch]}}
            Университетский курсовой проект, на данный момент ещё не завершён, планируется довольно значительный рефакторинг кода и изменение некоторых частей архитектуры.
            (в репозитории dev ветка довольно сильно опережает main ветку) \newline
            C++
            \item Клон игры Астероиды \newline \footnotesize \clink{\href{https://asmorodinov.github.io}{[asmorodinov.github.io]}} \newline
            \footnotesize \clink{\href{https://github.com/asmorodinov/asmorodinov.github.io}{[github.com/asmorodinov/asmorodinov.github.io]}}
            Школьный проект, писался на уроках информатики в 11 классе за пару недель. \newline
            Javascript
            \item Случайная генерация разбиения ромба на домино (random diamond tilings) \newline \footnotesize \clink{\href{https://jsfiddle.net/asmorodinov/afzqwmht/213/show}{[jsfiddle.net/asmorodinov/afzqwmht/213/show]}} \newline
            Просто реализация одного алгоритма на js, проект был написан под впечатлением от достаточно интересного математического видео по данной теме.
            Javascript
            
            \item 3d игра в жанре roguelike от первого лица \newline 
            Достаточно объёмный проект, написан на c++ без использования готовых игровых движков. \newline
            Игра ещё далека от завершения, код пока что не доступен, но возможно появится позже. \newline 
            C++
        \end{itemize}
    }
}

\end{tabularx}
\end{center}
\end{document}
